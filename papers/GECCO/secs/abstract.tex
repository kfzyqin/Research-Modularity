\begin{abstract}
A long-standing biological question is how organisms can quickly adapt themselves to new environments that are constantly changing, which is called evolvability. A key aspect to understand evolvability is to know the origin of modularity. In the computational biology, one prevalent theory argues that gene specialization drives modularity. Our experiments indicated that there existed an inconsistency between this theory and observations in biology, regarding the dominant status of modular structures on evolvability. Subsequent experiments also indicated that networks with high fitness could be converted into modular structures by removing inter-module connections while their performance improved. Further experiments demonstrated that more fluctuant evolving landscape can also promote a higher level of network modularity, compared with a more fixed landscape in which networks search for optimums. We also showed that more modular structures may require fewer connections. Therefore, modular networks may evolve towards structures that require a fewer total number of edges.
\end{abstract}